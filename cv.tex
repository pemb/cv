%%%%%%%%%%%%%%%%%%%%%%%%%%%%%%%%%%%%%%%%%
% Plasmati Graduate CV
% LaTeX Template
% Version 1.0 (24/3/13)
% 
% This template has been downloaded from:
% http://www.LaTeXTemplates.com
% 
% Original author:
% Alessandro Plasmati (alessandro.plasmati@gmail.com)
% 
% License:
% CC BY-NC-SA 3.0 (http://creativecommons.org/licenses/by-nc-sa/3.0/)
% 
% Important note:
% This template needs to be compiled with XeLaTeX.
% The main document font is called Fontin and can be downloaded for free
% from here: http://www.exljbris.com/fontin.html
% 
%%%%%%%%%%%%%%%%%%%%%%%%%%%%%%%%%%%%%%%%% 

% ----------------------------------------------------------------------------------------
%	PACKAGES AND OTHER DOCUMENT CONFIGURATIONS
% ----------------------------------------------------------------------------------------

\documentclass[a4paper,10pt]{article} % Default font size and paper size

\usepackage{fontspec} % For loading fonts
\defaultfontfeatures{Mapping=tex-text}
\setmainfont[SmallCapsFont = Gill Sans Light]
{Gill Sans Light} % Main document font

\usepackage{xunicode,xltxtra,url,parskip} % Formatting packages

\usepackage[usenames,dvipsnames]{xcolor} % Required for specifying custom colors

\usepackage[big]{layaureo} % Margin formatting of the A4 page, an alternative to layaureo can be \usepackage{fullpage}

% To reduce the height of the top margin uncomment: \addtolength{\voffset}{-1.3cm}

\usepackage[brazilian]{babel}
% \usepackage[utf8]{inputenc}
% \usepackage[T1]{fontenc}


\usepackage{hyperref} % Required for adding links	and customizing them
\definecolor{linkcolour}{rgb}{0,0.2,0.6} % Link color
\hypersetup{colorlinks,breaklinks,urlcolor=linkcolour,linkcolor=linkcolour} % Set link colors throughout the document

\usepackage{titlesec} % Used to customize the \section command
\titleformat{\section}{\Large\scshape\raggedright}{}{0em}{}[\titlerule] % Text formatting of sections
\titlespacing{\section}{0pt}{3pt}{3pt} % Spacing around sections



\usepackage{fontspec}
\makeatletter
\newlength\fake@f
\newlength\fake@c
\def\fakesc#1{%
  \begingroup%
  \xdef\fake@name{\csname\curr@fontshape/\f@size\endcsname}%
  \fontsize{\fontdimen8\fake@name}{\baselineskip}\selectfont%
  \uppercase{#1}%
  \endgroup%
}
\makeatother

\renewcommand{\textsc}{\fakesc}

\begin{document}

\pagestyle{empty} % Removes page numbering

\font\fb=''[cmr10]'' % Change the font of the \LaTeX command under the skills section

% ----------------------------------------------------------------------------------------
%	NAME AND CONTACT INFORMATION
% ----------------------------------------------------------------------------------------

\par{\centering{\Huge Pedro \textsc{Brito}}\bigskip\par} % Your name

\section{Dados Pessoais}

\begin{tabular}{rl}
  \textsc{Nome Completo:} & Pedro Emílio Machado de Brito\\
  %\textsc{Local e Data de Nascimento:} & Belo Horizonte, Minas Gerais | 1994-06-13 \\
  \textsc{Endereço:} & Rua Engenheiro Humberto Soares Camargo, 202\\
  & Cidade Universitária, Campinas -- SP \\
  \textsc{Telefone:} & (19) 99873-0963\\
  \textsc{E-mail:} & \href{mailto:pedroembrito@gmail.com}{pedroembrito@gmail.com}
\end{tabular}

% ----------------------------------------------------------------------------------------
%	WORK EXPERIENCE 
% ----------------------------------------------------------------------------------------

\section{Experiências}

\begin{tabular}{r|p{11cm}}
  \emph{Presente} & \textbf{Presidente} no Centro Acadêmico da Computação \\
 \textsc{Novembro de 2013} & \footnotesize{Diversas tarefas administrativas e de liderança.}\\
\multicolumn{2}{c}{} \\

  \emph{Novembro de 2013} & \textbf{Coordenador Tecnológico} no Centro Acadêmico da Computação \\
 \textsc{Outubro de 2012} & \footnotesize{Administrar um servidor web, implantação de um sistema de vigilância.}\\
\multicolumn{2}{c}{} \\

\emph{Dezembro de 2013} & \textbf{Bolsista PAD} da disciplina MC202 (Estruturas de Dados)\\
\textsc{Agosto de 2013} &  \emph{Prof. Guilherme P. Telles}\\
& \footnotesize {Auxiliar os alunos na codificação dos laboratórios, administrar uma lista de e-mails da disciplina, avaliar laboratórios de duas turmas.}\\
\multicolumn{2}{c}{}\\

\end{tabular}


% ----------------------------------------------------------------------------------------
%	EDUCATION
% ----------------------------------------------------------------------------------------

\section{Formação}

\begin{tabular}{r|l}	

  \textsc{2016} & Graduação em \textsc{}\textbf{Engenharia de Computação}, \small\emph{Sistemas de Computação}\\
  & \normalsize\textsc{Universidade Estadual de Campinas}\\
\multicolumn{2}{c}{}\\
  % ------------------------------------------------

  \textsc{2011} & \textbf{Ensino Médio}\\ & \textsc{Colégio Marista Diocesano}
\end{tabular}


% ----------------------------------------------------------------------------------------
%	Línguas
% ----------------------------------------------------------------------------------------

\section{Idiomas}

\begin{tabular}{rl}
  \textbf{Português:} & Nativo.\\
  \textbf{Inglês:} & Fluente para leitura e escrita, intermediário para conversação.\\
\end{tabular}

% ----------------------------------------------------------------------------------------
%       Competências
% ----------------------------------------------------------------------------------------

\section{Competências}

\begin{tabular}{rl}
  Conhecimentos básicos: & C++, Java, Python, \setmainfont{cmr10}{\fb \LaTeX} \setmainfont[SmallCapsFont=Fontin SmallCaps]{Fontin Sans}
  \\

  Conhecimentos intermediários: & C, GNU/Linux \\
\end{tabular}


\end{document}
